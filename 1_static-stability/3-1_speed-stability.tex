\section{Longitudinal Static Stability}
\subsection{Speed Stability}
For horizontal speed stability, an increase in $V_\infty$ must result in reduction of net force in the direction of velocity,

\begin{equation}
   X = Thrust - Drag \rightarrow \frac{\partial X}{\partial V_\infty} < 0
 \end{equation}

which only holds true if the aircraft is flying above the minimum drag speed. (from Aircraft Performance)

For vertical speed stability, downward velocity, ($W$), must result in reduction of net downward forces,
\begin{equation}
  Z = Weight - Lift \rightarrow \frac{\partial Z}{\partial W} < 0
\end{equation}
The aircraft is always statically stable in this regard.

\subsection{Pitch Stability}
An increase in AoA must result in a negative pitching moment (nose down).
\begin{equation}
  \frac{C_M}{\alpha} < 0 \rightarrow \frac{d C_M / d\alpha}{d C_L/d\alpha} = \frac{dC_M}{dC_L} < 0
\end{equation}

\subsection{Stick-Fixed Static Stability}
Starting from equation 1.2.20c and differentiating wrt. $C_L$,

\begin{subequations}
  \begin{equation}
    \frac{d\CMOW}{d C_L} = 0\qquad,\qquad\frac{\partial \CLH}{\partial C_L} = \frac{\partial \CLH}{\partial \alpha}\times\frac{\partial \alpha}{\partial C_L} = \frac{a_H}{a}\left(1 - \frac{d\epsilon}{d\alpha}\right)
  \end{equation}
  \begin{equation}
    \frac{dC_M}{dC_L} = (\xbcg - \xbw) - \overline{V}_H\frac{a_H}{a}\left(1 - \frac{d\epsilon}{d\alpha}\right) < 0
  \end{equation}
\end{subequations}

Therefore, stick-fixed static stability depends on
\begin{enumerate}
  \item the relative positions of CG and the aerodynamic surface
  \item the wing and tailplane lift curve slopes
  \item the wing and tailplane surface areas
\end{enumerate}

Considering a neutrally stable case, we get,
\begin{equation}
  \frac{dC_M}{dC_L} = 0 \rightarrow \xbnp = \xbw + \overline{V}_H\frac{a_H}{a}\left(1 - \frac{d\epsilon}{d\alpha}\right)
\end{equation}
Therefore,
\begin{equation}
  \frac{dC_M}{dC_L} = \xbcg - \xbnp
\end{equation}
and for stability,
\begin{equation}
  \xbcg - \xbnp < 0 \rightarrow \xbcg < \xbnp
\end{equation}
The aircraft's level of stability, is defined in terms of its static margin,
\begin{equation}
  K_n = \xbnp - \xbcg
\end{equation}
Higher the $K_n$, the larger the restorative pitching moment, leading to a very stable but sluggish aircraft and vice versa.

\subsection{Stick-Free Static Stability}
Similarly, differentiating equation 1.2.20c wrt. $C_L$, gives
\begin{equation}
  \frac{dC_M}{dC_L} = (\xbcg - \xbw) - \overline{V}_H\left[\frac{a_H}{a}\left(1 - \frac{d\epsilon}{d\alpha}\right)+a_E\frac{d\delta_E}{dC_L}\right] < 0
\end{equation}
as $\nicefrac{d\delta_E}{dC_L} \neq = 0$ in this case because the elevator is free to move (but trim tab is still fixed).

But that also means that we can consider the elevator hinge moment to determine equilibrium.

\begin{equation}
  H = qS_e\bar{c}_eC_H
\end{equation}
Then, $C_H$ is defined as 
\begin{subequations}
  \begin{align}
    C_H &= b_0 + \frac{dC_H}{d\alpha_H}\alpha_H + \frac{dC_H}{d\delta_E}\delta_E + \frac{dC_H}{d\delta_T}\delta_T\\
    &= b_0 + b_H\alpha_H + b_E\delta_E + b_T\delta_T
  \end{align}
\end{subequations}
Substituting in equation 1.2.10,
\begin{equation}
  C_H = b_0 + b_H\left[\alpha\left(1- \frac{d\epsilon}{d\alpha}\right)+i_H - \epsilon_0\right] + b_E\delta_E + b_T\delta_T
\end{equation}
After much masterful rearranging with $C_H = 0$,
\begin{equation}
  \delta E = -\frac{b_H}{b_E}\left[ \alpha\left(1- \frac{d\epsilon}{d\alpha}\right)+i_H - \epsilon_0 \right] - \frac{b_T}{b_E}\delta_T - \frac{b_0}{b_E}
\end{equation}
Differentiating
\begin{equation}
  \frac{d\delta_E}{dC_L} = \frac{d\delta_E}{d\alpha}\times\frac{dC_L}{\alpha} = -\frac{1}{a}\frac{b_H}{b_E}\left(1- \frac{d\epsilon}{d\alpha}\right)
\end{equation}
Finally, bringing it full circle by substituting this into equation 1.3.9 and simplifying to get
\begin{equation}
  \frac{dC_M}{dC_L} = (\xbcg - \xbw) - \overline{V}_H\frac{a_H}{a}\left(1 - \frac{d\epsilon}{d\alpha}\right)\left[1-\frac{a_E}{a_H}\frac{b_H}{b_E}\right]
\end{equation}
Similar to the stick-fixed case, the \textit{stick-free} neutral point and static margins are
\begin{subequations}
  \begin{equation}
    \bar{x}'_{np} = \xbw + \overline{V}_H\frac{a_H}{a}\left(1 - \frac{d\epsilon}{d\alpha}\right)\left[1-\frac{a_E}{a_H}\frac{b_H}{b_E}\right]
  \end{equation}
  \begin{equation}
    K'_n = \bar{x}'_{np} - \xbcg
  \end{equation}
\end{subequations}
Also because $a_h,a_E > 0 $ and $ b_h,b_E < 0$ in most cases,
\begin{equation}
  \left[1-\frac{a_E}{a_H}\frac{b_H}{b_E}\right] \leq 1 \rightarrow \bar{x}'_{np} \leq \xbnp \rightarrow K'_n \leq K_n
\end{equation}
if the aircraft is statically stable in the stick-free case, it also stable in the stick-fixed case. At also means that, if $b_H/b_E$ gets large, the free elevator could reduce or reverse the stabilising effect of the tailplane.
