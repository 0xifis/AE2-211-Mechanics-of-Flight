\subsection{Flight in Equilibrium}
Now lets consider the aircraft with the wing and tail as 1.
\begin{equation}
  L = L_W + L_H \rightarrow \CL = \CLW + \frac{S_H}{S_{ref}}\CLH
\end{equation}
After expanding equation 1.2.14, using equations 1.2.13 and 1.2.2b, we get the lift curve slope for the whole aircraft,
\begin{equation}
  a = \frac{dC_L}{d\alpha} = a_W + \frac{S_H}{S_{ref}}(1-\frac{d\epsilon}{d\alpha})a_H
\end{equation}
Similarly, pitching moment is
\begin{equation}
  M = M_{0W} + (\xcg - \xw)L_W + M_{0H} + (\xcg-\xH)L_H
\end{equation}
\begin{equation}
  C_M = \CMOW + \frac{(\xcg - \xw)}{\bar{c}}\CLW + \frac{S_H}{S_{ref}}\left(\frac{\bar{c}_H}{\bar{c}}\CMOH + \frac{\xcg -\xH}{\bar{c}_W}\CLH\right)
\end{equation}
Disregarding zero-lift moments contributed by trim and elevator tabs and going crazy with the expansion,
\begin{equation}
  C_M = \CMOW + \frac{(\xcg - \xw)}{\bar{c}}(\CLOW + a_W\alpha) + \frac{S_H}{S_{ref}}\frac{\xcg -\xH}{\bar{c}_W}\left[\CLOH + a_H(1-\frac{d\epsilon}{d\alpha})\alpha +a_E\delta_E+a_T\delta_T\right]
\end{equation}
From equation 1.2.14, we have $\CLW = C_L - \frac{S_H}{S_{ref}}\CLH$. Therefore, equation 1.2.17 can be expressed in terms of total aircraft $C_L$ and tailplane parameters.
\begin{align}
  C_M &= \CMOW + \frac{(\xcg - \xw)}{\bar{c}}C_L + \frac{S_H}{S_{ref}}\left[\frac{\xcg -\xH}{\bar{c}_W}\CLH - \frac{\xcg -\xw}{\bar{c}_W}\CLH \right] \\
  &= \CMOW + \frac{(\xcg - \xw)}{\bar{c}}C_L + \overline{V}_H\CLH
\end{align}

where the tail volume coefficient is,
\begin{equation}
  \overline{V}_H = \frac{S_H (\xcg - \xH)}{S_{ref}\bar{c}} = \frac{S_Hl_H}{S_{ref}\bar{c}}
\end{equation}

Going full circle, in order to achieve equilibrium, the following conditions must be met.
\begin{subequations}
  \begin{equation}
    C_L = \frac{W}{\nicefrac{1}{2}\rho V^2_\infty S_{ref}}
  \end{equation}
  \begin{equation}
    C_M = 0
  \end{equation}
\end{subequations}


