\section{Direction Static Stability}
Given a sideslip angle, $\beta$, for directional stability, the aircraft must have a natural tendency to develop a yawing moment, $N$, about the CG. ($+ve$ in CW from the top)
\begin{equation}
  \frac{dN}{d\beta} > 0
\end{equation}
for directional stability. The vertical stabiliser is the main source of yawing moment.
\begin{subequations}
  \begin{equation}
    N = (x_V - \xcg)L_V
  \end{equation}
  \begin{equation}
    L_V = qS_VC_{L_V} = qS_Va_V\beta
  \end{equation}
\end{subequations}
Since yawing moment is often represented non-dimensionally as 
\begin{equation}
  C_N = \frac{N}{qS_{ref}b}
\end{equation}
Therefore for the vertical stabiliser's contribution to directional stability,
\begin{equation}
  \frac{dN}{d\beta}_V = \frac{S_V(x_V-\xcg)}{S_{ref}b}a_V= \overline{V}_Va_V > 0
\end{equation}
Do note, that the fuselage contributes substantial destabilising effect so typically, $\overline{V}_Va_V$ has to be between $0.6$ and $0.9$.
